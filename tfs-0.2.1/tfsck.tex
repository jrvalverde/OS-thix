	 
\documentstyle[boxedminipage,a4wide,epsf,12pt]{article}

\def\a#1{
	{\bf #1}
}
\def\bi{
	\hspace*{0.0cm}
}
\def\bia{
	\hspace*{0.6cm}
}
\def\bib{
	\hspace*{1.2cm}
}
\def\bic{
	\hspace*{1.8cm}
}
\def\bid{
	\hspace*{2.4cm}
}
\def\bie{
	\hspace*{3.0cm}
}
\def\bif{
	\hspace*{3.6cm}
}
\def\big{
	\hspace*{4.2cm}
}
\def\bih{
	\hspace*{4.8cm}
}

\def\thefootnote{\fnsymbol{footnote}}

\title{\bf Repararea Sistemelor de Fisiere Thix}

\author{{\em Tudor Hulubei}\\
        Facultatea de Automatica si Calculatoare\\
	Universitatea Politehnica Bucuresti\\
        Romania}

\date{3 Februarie, 1996}

\begin{document}

\maketitle

\bibliographystyle{plain}

\large

\setlength{\parskip}{7pt plus 2pt minus 1pt}


\section{Introducere}


Acest proiect prezinta implementarea utilitarului de reparare a
sistemelor de fisiere {\bf tfs} ({\bf Thix File System}) utilizate de
kernelul de Thix.  Repararea sistemului de fisiere este o operatie
delicata si care poate duce la distrugerea partiala sau totala a
intregului sistem de fisiere daca nu este efectuata corect, dar este
de asemenea o operatie extrem de utila in faza de dezvoltare a unui
nucleu de sistem de operare.  Pentru o mai buna intelegere a
algoritmilor folositi voi incepe prin a descrie structura sistemului
de fisiere tfs, apoi voi prezenta implementarea curenta a utilitarului
{\bf fsck}.


\section{Structura Sistemului de Fisiere}


Ca multe dintre sistemele de fisiere moderne, {\bf tfs} este organizat
in clustere.  Aceasta organizare permite implementarea unor algoritmi
eficienti de optimizare a accesului la disk, incercand sa impuna in
masura posibilului alocarea blocurilor si a inodului unui fisier
intr-un anumit cluster.  Acest lucru permite atat evitarea
fragmentarii cat si minimizarea timpului de pozitionare a capetelor
discului ({\bf seek time}).

Primul bloc din {\bf tfs} este blocul de boot.  El contine prima parte
a incarcatorul sistemului de operare (first stage boot loader).  Acest
prim incarcator de mici dimensiuni are rolul de a porni un al doilea
incarcator (second stage boot loader), un program complex ce cunoaste
structura sistemului de fisiere si este capabil sa localizeze pe disc
blocurile ce alcatuiesc nucleul sistemului de operare (nucleul este un
fisier obisnuit), sa-l incarce si sa-i dea controlul.  Spatiul alocat
celui de-al doilea incarcator este rezervat la crearea systemului de
fisiere.

Imediat dupa zona rezervata celui de-al doilea incarcator se gaseste
zona rezervata blocurilor defecte.  Ca si in cazul incarcatorului
sistemului de operare, dimensiunea acestei zone se stabileste la
crearea sistemului de fisiere.

Restul discului este folosit pentru date.  Spatiul ramas este impartit
intr-un numar de clustere, fiecare dintre acestea fiind de fapt un
sistem de fisiere in miniatura.

Un cluster este alctuit din:


\begin{itemize}

\item o copie a superblocului ce este actualizata de sistemul de
operare la fiecare actualizare a superblocului (la executia apelului
sistem {\bf sync()} si la demontarea sistemului de fisiere - {\bf
umount()}).

\item o tabela de biti continand informatii referitoare la
alocarea inodurilor din clusterul respectiv.

\item o tabela de biti continand informatii referitoare la
alocarea blocurilor de date din clusterul respectiv.

\item o zona ce contine inodurile propriu-zise.

\item o zona ce contine blocurile de date.

\end{itemize}

Tabelele de biti pentru inoduri si blocuri ocupa un bloc (1024 bytes),
putand referi maxim 1024*8=8192 inoduri sau blocuri, cate un bit
pentru fiecare, specificand starea inodului sau blocului respectiv:
alocat sau liber.

Aceasta abordare permite alocarea blocurilor si inodurilor dintr-un
anumit cluster pana cand ele se epuizeaza, moment in care se trece la
un alt cluster, in general cel care contine cel mai mare numar de
inoduri si blocuri libere.  De asemenea se pot aloca pentru fisiere
blocuri care se gasesc pe disc in zone apropiate: odata alocat un bloc
pentru un fisier, se poate cauta in tabela de alocare a blocurilor cel
mai apropiat bloc disponibil, pentru a micsora pe cat posibil
fragmentarea.


\section{Utilitarul fsck}


\subsection{Verificarea Superblocului}


Prima faza a repararii sistemului de fisiere consta in verificarea
consistentei superblocului.  Pentru aceasta se verifica initial
numarul magic si suma de control a superblocului.  Acestea trebuie sa
fie corecte, altminteri {\bf fsck} va refuza sa continue.  Astfel {\bf
fsck} nu va incerca sa repare sisteme de fisiere {\bf non-tfs}.  Chiar
daca exista inconsistente in structura sistemului de fisiere si
versiunea curenta a superblocului nu reflecta structura curenta a
datelor de pe disc, suma de control a superblocului si numarul magic
trebuie sa fie corecte (numarul magic este setat la crearea sistemului
de fisiere iar suma de control la fiecare actualizare a
superblocului).

In continuare se verifica consistenta campurilor din superblock, cum
ar fi dimensiunea intrarilor in director, numarul total de blocuri si
inoduri din sistemul de fisiere, etc.  Eventualele erori sunt
raportate, corectiile urmand sa se faca ulterior in momentul in care
intregul sistem de fisiere va fi parcurs si se vor cunoaste valorile
corecte.


\subsection{Verificarea Inodurilor si a Blocurilor}


Al doilea pas consta in verificarea consistentei informatiei din
tabelele de alocare ale inodurilor si blocurilor din fiecare cluster.
Toate inodurile marcate {\em liber} in tabela de alocare trebuie sa
contina 0 in campul {\bf di\_mode} si nu trebuie sa aiba link-uri.

Inodurile ce sunt alocate trebuie sa contina un tip valid de fisier in
campul {\bf di\_mode}, altminteri {\bf fsck} va cere o valoare de la
tastatura.  Valoarea implicita este {\bf S\_IFREG} (fisier obisnuit).
De asemenea, un inod alocat trebuie sa aiba cel putin un link.

Odata efectuate aceste teste, pentru fiecare inod folosit se
calculeaza numarul de blocuri utilizate, parcurgand atat lista de
blocuri directe din inod cat si blocurile indirecte.  Numarul total de
blocuri folosite trebuie sa corespunda campului {\bf di\_blocks} din
inod.  In caz contrar, campul {\bf di\_blocks} este corectat.

Pe parcursul verificarii inodurilor se construiesc in memorie tabelele
de alocare a blocurilor.  Pentru fiecare bloc ce apartine unui inod
(este alocat) se seteaza bitul de {\em alocat} in tabela de alocare a
blocurilor din clusterul corespunzator pentru a putea compara ulterior
tabelele calculate cu cele ce se gasesc pe disc si a se putea face
eventualele corectii.

Daca erorile detectate in pasul 2 sunt severe si nu pot fi in
intregime corectate pe parcurs, {\bf fsck}-ul se va opri.  In acest
caz se poate relua executia primelor doi pasi de oricate ori este
nevoie pentru a corecta toate erorile.  Numai dupa aceasta se va putea
trece la pasul 3.  Detectia unei erori duce la incrementarea
contorului global de erori, dar daca se poate face corectia, contorul
este decrementat in asa fel incat sa se poata trece la pasul 3.


\subsection{Verificarea Fisierelor si a Directoarelor}

In pasul 3 se realizeaza o parcurgere a sistemului de fisiere
orientata pe directoare/fisiere (spre deosebire de parcurgerea facuta
la pasul 2 orientata pe inoduri/blocuri).

Un director valid trebuie sa contina intrari pentru directoarele {\bf
.}  si {\bf ..}.  Aceste intrari trebuie sa contina numere de inod
valide, altminteri intrarea respectiva este stearsa.  In acest pas se
calculeaza de asemenea numarul de link-uri ale fiecarui inod, aceasta
informatie urmand a fi folosita in pasul urmator.


\subsection{Reverificarea Inodurilor}

In pasul 4 se verifica numarul de link-uri ale fiecarui inod.  Campul
din inod ce specifica numarul de link-uri trebuie sa fie egal cu cel
calculat la pasul 3 in timpul parcurgerii intregului sistem de
fisiere.  Daca apar diferente, campul din inod este setat la valoarea
calculata.


\subsection{Reluarea Verificarii}


{\bf fsck} nu va reusi intotdeauna sa corecteze sistemul de fisiere
intr-o singura rulare.  De exemplu, inconsistente detectate (si
eventual corectate) la pasul 3 pot duce la inconsistente detectabile
doar la pasul 2.  De exemplu, de abia in pasul 3 putem detecta faptul
ca un anumit inod (altminteri corect alocat si continand blocuri
valide) nu este referit din nici un director.  Inodul va fi marcat ca
{\em liber} ({\bf di\_mode} = 0) in pasul 3, fara insa a-l reparcurge
si dealoca blocurile ce-i apartin (deoarece aceasta operatie este
implementata in pasul 2).  Ideea este de a efectua o corectie treptata
si solida a structurii sistemului de fisiere si de evita pe cat
posibil interferentele intre pasi.  La un nou apel al utilitarului
{\bf fsck} se va detecta in pasul 2 faptul ca un inod nealocat contine
pointeri la blocuri alocate si blocurile corespunzatoare vor fi la
randul lor dealocate.

Din acest motiv, {\bf fsck} va intoarce codul 0 (ok) numai daca nu a
fost detectata nici o eroare.  Utilizand acest lucru, utilitarul {\bf
fsck} poate fi apelat intr-o bucla ce se termina numai atunci cand
{\bf fsck} intoarce 0 semnificand absenta erorilor in structura
sistemului de fisiere.

\end{document}
