\begin{slide}{}


{\small \bf \center 3. Virtual Memory Management\\
}

\vspace{2cm}

{\tiny Virtual Memory has been implemented taking advantage of the
flexible design of the Intel 80386 processors.

\vspace{1cm}


\begin{tabbing}
\hspace{1cm} - \= \kill
\hspace{1cm} - each process has a completely different address space,\\
\> build using a page directory and several page tables
\end{tabbing}

\begin{tabbing}
\hspace{1cm} - \= \kill
\hspace{1cm} - text pages are allways shared between different\\
\> instances of the same program
\end{tabbing}

\hspace{1cm} - data pages are shared as long as they are not modified

\begin{tabbing}
\hspace{1cm} - \= \kill
\hspace{1cm} - invalid memory references are always detected and the\\
\> appropriate signal (SIGSEGV) is sent to the offending process
\end{tabbing}

\begin{tabbing}
\hspace{1cm} - \= \kill
\hspace{1cm} - a swapper process is started each time the system\\
\> is low on memory and starts swapping out unused pages
\end{tabbing}

\hspace{1cm} - a memory allocation mechanism is provided inside the kernel

\begin{tabbing}
\hspace{1cm} - \= \kill
\hspace{1cm} - the virtual memory size can be increased using multiple\\
\> swap devices
\end{tabbing}

\hspace{1cm} - the kernel is able to get around the 80386 VM flaw

}

\end{slide}
