\begin{slide}{}


{\small \bf \center 4. The Thix File System\\
}

\vspace{2cm}

{\tiny {\em Thix} provides a native file system, designed to be fast
and fault tolerant.  The file system is divided into clusters, each
cluster containing blocks and inodes.  This approach has several
advantages over the original Unix System V implementation (a file
system using a linked list for block allocation):

\vspace{1cm}


\begin{tabbing}
\hspace{1cm} - \= \kill
\hspace{1cm} - the file system fragmentation can be controlled by\\
\> trying to allocate all the file blocks from one cluster.
\end{tabbing}

\begin{tabbing}
\hspace{1cm} - \= \kill
\hspace{1cm} - the seek time can be optimized; by allocating blocks and\\
\> inodes from the same cluster, files will not be scattered all\\
\> over the file system
\end{tabbing}

\begin{tabbing}
\hspace{1cm} - \= \kill
\hspace{1cm} - there is no need to parse the entire inode area in\\
\> order to find free inodes; a new inode can be found by\\
\> searching the current cluster's inodes bitmap
\end{tabbing}

\begin{tabbing}
\hspace{1cm} - \= \kill
\hspace{1cm} - the file system is much more stable than in the\\
\> original implementation
\end{tabbing}

\hspace{1cm} - a dynamically expandable buffer cache is provided

\hspace{1cm} - a memory allocation mechanism is provided inside the kernel

\begin{tabbing}
\hspace{1cm} - \= \kill
\hspace{1cm} - the file system superblock has backups at the begining\\
\> of each cluster
\end{tabbing}

\begin{tabbing}
\hspace{1cm} - \= \kill
\hspace{1cm} - a fsck utility is provided; it is able to repair a really\\
\> damaged file system with minimum data loss
\end{tabbing}

}

\end{slide}
